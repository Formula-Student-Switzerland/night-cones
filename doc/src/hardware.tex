\section{Hardware}

\subsection{Overview}

\subsection{Controller}
An ESP8266 on a module is selected as microcontroller. The module is used as it already contains the matching network and either an antenna or anantenna connector. 
The usage of an ESP32 was considered as well but finally rejected due to the higher cost. 

Three options are considered for the module: ESP-01S, ESP-07S and ESP-12F. \autoref{tab_ESP_modules} lists the modules and their most important spacifications. 

\begin{table}[h!]
    \centering
    \begin{zebratabular}{llll}
    \rowcolor{gray} 
    Module  & GPIO  & ADC pin   & Antenna \\
    ESP-01S & 4     & No        & internal \\
    ESP-07S & 11    & No        & internal \\
    ESP-12F & 17    & No        & internal \\
    \end{zebratabular}
    \caption{Overview of ESP6288 modules}
    \label{tab_ESP_modules}
\end{table}

It is possible to solder any of the modules that is listed in \autoref{tab_ESP_modules} on the PCB. The ESP-07S can be soldered on the same footprint as the ESP-12F. The ESP-01S footprint could be fitted inside the ESP-12F footprint. Special car must be taken in that case, as the ESP-07S has an exposed pad. Only the GND pin of the ESP-01S is allowed to touch the area of the exposed pad to prevent short-circuits when the EXP-07S is mounted. 

The ESP module is powered from the \SI{3.3}{\volt} supply. For details about the \SI{3.3}{\volt} supply, see \autoref{sec_power_3V3}, \nameref{sec_power_3V3}. 

The EN pin and the RST pin are pulled to \SI{3.3}{\volt} with a pull-up resistor each. The GPIO pinout is listed in \autoref{tab_ESP_pinout}. 

\begin{table}[h!]
    \centering
    \begin{zebratabular}{lllll}
    \rowcolor{gray} 
    Signal      & ESP-01S   & ESP-07S   & ESP-12F   & Description \\
    DATA        & GPIO0     & GPIO15    & GPIO15    & LED data \\
    KILL        & GPIO2     & GPIO13    & GPIO13    & Turn-off command \\
    HALL        & N/A       & GPIO14    & GPIO14    & Presence detection from hall sensor \\
    VBAT\_MON   & N/A       & ADC       & ADC       & Battery voltage monitor \\
    \end{zebratabular}
    \caption{ESP module pinout}
    \label{tab_ESP_pinout}
\end{table}

\subsection{LED}

\subsection{Failsafe}

\subsection{Charger}

\subsection{Battery Management System}

\subsection{On/Off controller}

\subsection{Converter \SI{5.2}{\volt}}

\subsection{Converter \SI{3.3}{\volt}}
\label{sec_power_3V3}

